\documentclass[a4paper]{article}

\usepackage[inner=2.0cm,outer=2.0cm,top=2.5cm,bottom=2.5cm]{geometry}
\usepackage[colorlinks=true, urlcolor=blue,  linkcolor=blue, citecolor=blue]{hyperref}
\usepackage{amsthm}
\usepackage{amsmath}
\usepackage{amssymb}
\usepackage{graphicx}
\usepackage[margin=4cm]{caption}

\setlength\parindent{0pt}
\setlength\parskip{0.5em}

\theoremstyle{definition}
\newtheorem{question}{Question}

% hint for questions
\newcommand{\qhint}[1]{{%
    \begin{center}
        \scriptsize \textbf{Hint:} #1
    \end{center}}}
% marks for each question
\newcommand{\qmarks}[1]{{\hspace*{\fill} [#1 marks]}}

\newcommand{\CommentNP}[1]{\textcolor{blue}{(Nikhil: #1)}}
\newcommand{\CommentAH}[1]{\textcolor{red}{(Aditya: #1)}}
\newcommand{\CommentAC}[1]{\textcolor{green}{(Ashish: #1)}}


\begin{document}
\begin{center}
    {\Large CS 201: Discrete Mathematics}\\
    {\Large Quiz 1}\\
    {Maximum Marks: 50}
\end{center}
\vspace*{4mm}

\textbf{Instructions}
\begin{enumerate}
    \itemsep0em
    \item Any result stated or proved in the class or tutorial can be used without repeating the proof.
    \item State all references used other than the class slides, problem sets and the textbooks.
\end{enumerate}
\vspace{2em}

\begin{question}
    The Minesweeper game consists of a minefield represented by a rectangular grid where each cell can be cleared or uncovered by clicking on it.
    If a cell containing a mine is clicked, the game ends.
    If the cell does not contain a mine, a number between 0 and 8 appears indicating the number of adjacent cells (including diagonally-adjacent) containing mines.
    An example of a game state is given in Figure~\ref{fig:minesweeper_game_state}.

    \begin{figure}[h]
	\centering
	\begin{tabular}{|c|c|c|c|c|c|c|c|}
	    \hline
	    0 & 0 & 1 & & & & & \\
	    \hline
	    0 & 1 & 2 & & & & & \\
	    \hline
	    0 & 1 & & & & & & \\
	    \hline
	    0 & 2 & 3 & & & & & \\
	    \hline
	    0 & 1 & & & 2 & 2 & 1 & \\
	    \hline
	    0 & 2 & 3 & 3 & 1 & 1 & & \\
	    \hline
	    0 & 1 & & 1 & 0 & 1 & 2 & \\
	    \hline
	    0 & 1 & 1 & 1 & 0 & 0 & 1 & \\
	    \hline
	\end{tabular}
	\caption{A sample minesweeper game state. Empty cells correspond to those cells that have not been clicked.}%
	\label{fig:minesweeper_game_state}
    \end{figure}

    In the following questions we try to capture some aspects of the Minesweeper game through formal logic.
    You are required to first define a set of predicates using which the following questions can be answered i.e., these predicates constitute a formal language for Minesweeper.

    \begin{enumerate}
	    \item Clearly define and state a set of predicates using which the following statements can be expressed as compound propositions.
		\begin{enumerate}
			\item There are exactly $n$ mines on the minefield.
			\item If a cell contains the number $1$, then there is exactly one mine in the adjacent cells.

			\qmarks{3 + 2 = 5}
		\end{enumerate}

	    \item Using the predicates defined in the previous sub-part, formally show that there must be a mine in the cell $(2, 2)$ (indexing from $0$) for the game state shown in Figure~\ref{fig:minesweeper_game_state}.

		\qmarks{5}
    \end{enumerate}
\end{question}

\begin{question}\
    \begin{enumerate}
	\item Use \textit{rules of inference} to prove the following arguments are valid:

		  $\begin{array}{rl}
			& \exists x \; (P(x) \; \land \; \lnot Q(x)) \\
			& \forall x \; (\lnot R(x) \; \rightarrow \; \lnot P(x)) \\
			& \forall x \; (R(x) \; \rightarrow \; S(x)) \\
			\cline{2-2}
            \therefore & \exists x \; (S(x) \; \land \; \lnot Q(x)) \\
		  \end{array}$

	  \item Use \textit{propositional resolution} to prove the following:

		  $\begin{array}{rl}
			& \lnot(p \land \lnot q) \lor \lnot(\lnot s \land \lnot t) \\
			& \lnot (t \lor q) \\
			& u \rightarrow (\lnot t \rightarrow (\lnot s \land p))\\
			\cline{2-2}
			\therefore & \lnot u \\
		  \end{array}$
    \qmarks{2 + 3 = 5} % Incorrectly placed to fit into same page
    \end{enumerate}
\end{question}

\begin{question}
    Given $n$ elements, let $x_i, \; 1 \le i \le n$ be the propositional variable which is true if the $i$-th element is in the set $\mathcal{X}$.
    Similarly, let $y_i, \; 1 \le i \le n$ be true if the $i$-th element is in the set $\mathcal{Y}$.
    Give compound statements to express the following:

    \begin{enumerate}
	\item $S_1(k)$: There are at most $k$ elements in $\mathcal{X}$ where $k \ge 0$.
	\item $S_2(k)$: There are exactly $k$ elements in $\mathcal{X}$ where $k \ge 0$.
	\item $S_3$: Every element in $\mathcal{Y}$ is also in $\mathcal{X}$ but $\mathcal{Y} \neq \mathcal{X}$ i.e., $\mathcal{Y}$ is a proper subset of $\mathcal{X}$.
	\item $S_4$: There is exactly one common element between $\mathcal{X}$ and $\mathcal{Y}$.
    \end{enumerate}
    \qmarks{3 + 2 + 3 + 2 = 10}
\end{question}

\begin{question}\
    \begin{enumerate}
	\item Discuss the error in the following proof by strong induction for the statement $a^n = 1$, $\forall n \ge 0 \land n \in \mathbb{Z}$ and $\forall a \neq 0 \land a \in \mathbb{R}$.

	    Proof:
	    \begin{itemize}
		\item Let $P(n)$ be the predicate $a^i = 1$, $\forall i \; 0 \le i \le n$.
		\item \textbf{Base case}: $P(0)$ is clearly true since $a^0 = 1$, $\forall a \neq 0 \land a \in \mathbb{R}$.
		\item \textbf{Inductive hypothesis}: Let $P(k)$ be true.
		\item \textbf{Inductive step}: We have
		    \begin{equation*}
			a^{k+1} = \frac{a^k \cdot a^k}{a^{k-1}} = 1
		    \end{equation*}
		    where the second equality follows from the inductive hypothesis.
		    We have thus shown that $a^{k+1} = 1$ which when combined with $P(k)$ implies $P(k+1)$.
		    Therefore, from induction, $P(n)$ holds $\forall n \in \mathbb{Z}$.
	    \end{itemize}

        \item Prove that
            \begin{equation*}
                \frac{1}{2} \cdot \frac{3}{4} \ldots \frac{2n - 1}{2n} \le \frac{1}{\sqrt{3n}}
            \end{equation*}
    \end{enumerate}

    \qmarks{2 + 3 = 5}
\end{question}

\begin{question}\
    \begin{enumerate}
        \item If $\frac{3}{20}x^5 - \frac{1}{12}x^4 - \frac{1}{6}x^3 - \frac{1}{2}x^2
            + 2x - 1 \geq 0$, then show that $x \geq 0$.
        \item If $a, b, c, d, e$ are real numbers such that the equation $ax^2 + (c + b)x + e + d = 0$ has real roots greater than 1, show that the equation $ax^4 + bx^3 + cx^2 + dx + e$ has at least one real root.
    \end{enumerate}
    \qmarks{5 + 5 = 10}
\end{question}

\begin{figure}[b]
    \centering
    \includegraphics[scale=0.5]{../figures/house_simulation.png}
    \caption{Change in simulation state across two time steps}\label{fig:house_simulation}
\end{figure}
\begin{question}
    Alice is trying to understand the spread of COVID-{19} in a city through simulations.
    The city is represented by a $n \times n$ grid where each cell corresponds to a house.
    Each house can either be \textit{infected} or \textit{healthy} (assume for this question that all members of the house are either infected or healthy).
    Two houses are \textit{neighbours} if they share an edge on the grid i.e., each house has at most four neighbours.
    The simulation starts with a few of the houses initially infected and the infection spreads at every time step.
    A house is infected in the next time step if either
    \begin{itemize}
	\item The house was previously infected or
	\item The house has at least two neighbours that were already infected.
    \end{itemize}

    The change in simulation state for two time steps are are shown in Figure~\ref{fig:house_simulation}.
    Prove that if fewer than $n$ houses in the city are initially infected, the whole city will never be completely infected.

    \qmarks{10}
\end{question}

\end{document}
